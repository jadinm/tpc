Networks enable hosts to exchange information through a series of switches, routers and links. Some networks are over-provisioned, the network capacity is larger than the amount of data that the attached hosts can send or receive. Given economic constraints, almost all non-trivial networks, in particular the wide area ones, do not have enough capacity to handle all possible traffic patterns. They can carry some traffic patterns without any loss or queueing, but for most of the patterns, the hosts need to adjust their transmission rate to match the available network capacity.

Over the years, a wide range of techniques have been designed to enable the networks and the hosts to dynamically adjust. On the network side, a range of traffic engineering techniques have been proposed to enable network operators to control the amount of incoming traffic or redirect flows on less heavily used links. Example techniques include setting OSPF weights \cite{fortz2000internet}, using ATM, optical paths or MPLS \cite{awduche2002overview,xiao2000traffic} or mor recently leveraging Software Defined Networking \cite{jain2013b4} or Segment Routing \cite{filsfils2015segment}. These techniques typically operate on large aggregated flows and their reaction time is on the order of minutes or longer.

On a shorter timescale, endhosts have also adopted a range of congestion control techniques to adjust their transmission rate based on feedback received from the network. Many congestion control schemes have been included in the TCP protocol. They adjust the transmission rate based on the observed packet losses \cite{jacobson1988congestion,ha2008cubic}, delay variations \cite{brakmo1995tcp}, explicit network feedback \cite{ramakrishnan1999proposal} or a combination of different signals \cite{dong2015pcc,cardwell2017bbr}. Other networks have explored different types of feedback such as explicit rate \cite{kalyanaraman2000erica} or different forms of credit \cite{kung1995credit} in ATM networks. All these approaches consider the congestion control problem as a temporal one. All the packets belonging to a given flow follow the same path and they develop algorithms and heuristics to determine \textbf{when} the next packet should be transmitted and delay packets when the network is congested.

Researchers have explored other techniques to better utilize the available network capacity. Theoretical results showing that it is interesting to utilize the different paths that exist between a pair of hosts \cite{kelly2005stability,key2007multipath,wischik2008resource} have motivated the design of Multipath TCP \cite{raiciu2012hard} and its associated coupled congestion control scheme \cite{wischik2011design}. Other multipath protocols have also adopted similar solutions \cite{iyengar2006concurrent,de2017multipath}.

Although researchers showed the benefits that Multipath TCP could bring in datacenters \cite{raiciu2011improving}, operators preferred to stick with single path protocols such as TCP. Given that TCP uses a single path, researchers explored the possibility of spraying packets over different paths \cite{dixit2013impact} and techniques to reroute TCP connections when congestion is detected \cite{kabbani2014flowbender}. 

End-to-end approach have demonstrated their benefits to solve a wide range of network problems \cite{saltzer1984end}. In this paper, we propose to add a spatial dimension to congestion control schemes. More specifically, we enable a TCP sender to use a set of $n$ paths for each TCP connection. When a congestion event occurs, the sender can react by moving the TCP connection over another path from its set. This enables TCP senders to explore and use the different paths that reach the destination of each connection.

This paper is organised as follows. We first discuss in Section~\ref{section:paths} how networks can efficiently expose different paths to endhosts. Due to paper size limitation, we focus on using IPv6 Segment Routing \cite{previdi2017ipv6} in this paper, but our solutions are applicable to other techniques that enable endhosts to select network paths. Then in Section~\ref{section:ebpf}, we leverage eBPF to enable the Linux TCP stack to change the path of a connection upon detection of a congestion event. In Section~\ref{section:eval} we evaluate the performance of \pathchanger{} in simple enterprise networks that utilize a few parallel wide-area links. We compare our approach to existing solutions. In Section~\ref{section:topo} we use \pathchanger{} in emulated enterprise networks and demonstrate the ability to our solution to fully utilize the network even when only a few hosts communicate.